\section*{sleepy\-M\-U\-S\-T\-A\-C\-H\-E}

Doxygen \href{http://www.sleepymustache.com/documentation/html/index.html}{\tt Documentation} is available.

sleepy\-M\-U\-S\-T\-A\-C\-H\-E is a P\-H\-P framework that comes with solutions for everyday php challenges. Most of the functionality is optional and tries to be as minimalist as possible.

\subsection*{Functionality}

\subsubsection*{Core Functionality}

The core is the basic functions that modules build off of


\begin{DoxyItemize}
\item {\bfseries Debugging} -\/ Easily send debug information via the browser, email, or database
\end{DoxyItemize}


\begin{DoxyItemize}
\item {\bfseries Hooks} -\/ Assign functions to run at specific hook points throughout the loading process. This module is responsible for handling almost all functionality.
\end{DoxyItemize}


\begin{DoxyItemize}
\item {\bfseries Benchmarking} Lets you time the speed of functions
\end{DoxyItemize}


\begin{DoxyItemize}
\item {\bfseries Templating} -\/ Basic templating functionality lets you separate business logic from the view. It replaces placeholders like \char`\"{}\{\{ title \}\}\char`\"{} with data.
\end{DoxyItemize}

\subsubsection*{Default Modules}

Some modules are enabled by default. To disable the modules move them from the \char`\"{}enabled\char`\"{} folder and put them into the \char`\"{}disabled\char`\"{} folder.


\begin{DoxyItemize}
\item {\bfseries H\-T\-M\-L Compress} -\/ Compresses the output H\-T\-M\-L if the E\-N\-V variable is set to \char`\"{}\-L\-I\-V\-E\char`\"{}
\end{DoxyItemize}


\begin{DoxyItemize}
\item {\bfseries \hyperlink{class_c_s_s}{C\-S\-S} Compress} -\/ Compressed the output \hyperlink{class_c_s_s}{C\-S\-S} if the E\-N\-V variable is set to \char`\"{}\-L\-I\-V\-E\char`\"{}
\end{DoxyItemize}


\begin{DoxyItemize}
\item {\bfseries Head Inserter -\/ Joey Bomber} -\/ Allows you to insert H\-T\-M\-L to the bottom of the H\-E\-A\-D tag
\end{DoxyItemize}


\begin{DoxyItemize}
\item {\bfseries Robots Dev Hide -\/ Joey Bomber} -\/ If the site is not live, add the meta robots tag to omit from Google search
\end{DoxyItemize}


\begin{DoxyItemize}
\item {\bfseries \hyperlink{class_navigation}{Navigation}} -\/ Creates a U\-L that can be used for a navigation
\end{DoxyItemize}


\begin{DoxyItemize}
\item {\bfseries U\-R\-L Class} -\/ Adds a class based on the current page. /user/jaime will have a class added to the body that says \char`\"{}user-\/jaime-\/index\char`\"{}. Additionally /user would be \char`\"{}user-\/index\char`\"{}
\end{DoxyItemize}

\subsubsection*{Available Modules}

Most modules are disabled by default. To enable the modules move them from the into the \char`\"{}disabled\char`\"{} folder into \char`\"{}enabled\char`\"{} folder


\begin{DoxyItemize}
\item {\bfseries \hyperlink{class_c_s_v}{C\-S\-V}} -\/ C\-R\-U\-D class for \hyperlink{class_c_s_v}{C\-S\-V} files.
\end{DoxyItemize}


\begin{DoxyItemize}
\item {\bfseries \hyperlink{class_d_b}{D\-B}} -\/ Create, Read, Update, Delete (C\-R\-U\-D) functionality using P\-D\-O and my\-S\-Q\-L.
\end{DoxyItemize}


\begin{DoxyItemize}
\item {\bfseries \hyperlink{class_d_b}{D\-B} Grid} -\/ Turns a S\-Q\-L Select statement into a table with hook points
\end{DoxyItemize}


\begin{DoxyItemize}
\item {\bfseries File System Database} -\/ A basic database that uses flat files and J\-S\-O\-N documents.
\end{DoxyItemize}


\begin{DoxyItemize}
\item {\bfseries I\-P 2 Country} -\/ Uses the I\-P address to detect the country of origin.
\end{DoxyItemize}


\begin{DoxyItemize}
\item {\bfseries \hyperlink{class_mailer}{Mailer}} -\/ Provides basic email functionality with R\-F\-C email validation.
\end{DoxyItemize}


\begin{DoxyItemize}
\item {\bfseries Memcache} -\/ Improve performance by implementing memcaching of pages (10 second default)
\end{DoxyItemize}


\begin{DoxyItemize}
\item {\bfseries Mobile detection} -\/ Can detect mobile and tablet devices
\end{DoxyItemize}


\begin{DoxyItemize}
\item {\bfseries Robotalker} -\/ Class for interacting with the robotalker A\-P\-I
\end{DoxyItemize}


\begin{DoxyItemize}
\item {\bfseries Users} -\/ Basic user and roles functionality includes auth, roles, and permissions
\end{DoxyItemize}

\subsubsection*{Sample Modules}

These module have little-\/to-\/no practical use, but help demonstrate how to build simple modules with hook points.


\begin{DoxyItemize}
\item {\bfseries Timer} -\/ A module that replaces the \{\{ timer \}\} with the time it took the framework to generate the page.
\item {\bfseries Wizard Title} -\/ This module prepends an A\-S\-C\-I\-I wizard to the title of the page.
\end{DoxyItemize}

\subsection*{Getting Started}

There are a few globals you will want to set in the include/globals.\-php file.


\begin{DoxyItemize}
\item Setup debugging
\item Set Live site U\-R\-L
\item Set \hyperlink{class_d_b}{D\-B} credentials for live/stage/dev
\item Set Emailing info for live/stage/dev
\item Setup G\-A Account for live/state/dev
\end{DoxyItemize}

\subsection*{Sample Code}

\subsubsection*{Hooks}

The {\itshape Hooks} system is made up of {\itshape hook filters} and {\itshape hook actions}. {\itshape \hyperlink{class_hook}{Hook} actions} are points in the code where you can assign functions to run. For example, we can put a {\itshape hook action} after a record is saved to the database, then assign a function to the {\itshape hook action} that will send an email after the \hyperlink{class_d_b}{D\-B} update. \begin{DoxyVerb}           // Save to the database
           $db->save();

           // add a hook action
           Hook::addAction('record_saved');

           // Add a function to the hook action
           function send_email() {
                          // send an email saying a record was updated
           }

           Hook::doAction(
                          'record_saved',
                          'send_email'
           );
\end{DoxyVerb}


{\itshape \hyperlink{class_hook}{Hook} filters} are similar to {\itshape hook actions} but pass data as parameters to the functions that get assigned to the hook. After manipulating this data you should return the edited data back to the program. \begin{DoxyVerb}           // add a hook filter
           $content = Hook::addFilter('update_content', $_POST['content']);

           // Add a function to the hook filter
           function clean_html ($html) {
                          $c = htmlentities(trim($html), ENT_NOQUOTES, "UTF-8", false);
                          return $c;
           }

           Hook::applyFilter(
                          'update_content',
                          'clean_html'
           );
\end{DoxyVerb}


The {\itshape modules/enabled} directory provides a convenient location to put code that utilized the hooks system. Code inside of the {\itshape modules/enabled} directory are automatically added to the program at runtime.

\subsubsection*{Templating}

Templates reside inside the $\ast$'/templates/'$\ast$ folder and should end in a .tpl extension. The templating system works by using placeholders that later get replaced with text. The placeholders must have the following syntax\-: \begin{DoxyVerb}           {{ placeholder }}
\end{DoxyVerb}


To use a template you instantiate the template class passing in the template name. You then bind data to the placeholders and call the {\itshape \hyperlink{class_template_a2b8e3779f5bd8c38f70307574859bd36}{Template\-::show()}} method. \begin{DoxyVerb}           require_once('include/sleepy.php');

           $page = new Template('default');
           $page->bind('title', 'sleepyMUSTACHE');
           $page->bind('header', 'Hello world!');
           $page->show();
\end{DoxyVerb}


Here is the sample template file (templates/default.\-tpl) \begin{DoxyVerb}           <html>
                          <head>
                                         <title>{{ title }}</title>
                          </head>
                          <body>
                                         <h1>{{ header }}</h1>
                                         <p>This page has been viewed {{ hits }} times.</p>
                          </body>
           </html>
\end{DoxyVerb}


We added a $\ast$\{\{ hits \}\}$\ast$ placeholder in the template above. For this example, we want to replace the placeholder with the number of times this page was viewed. We can add that functionality using Hooks. \begin{DoxyVerb}           // filename: /modules/enabled/hit-counter/hits.php
           function hook_render_placeholder_hits() {
                          $hits = new FakeClass();

                          return $hits->getTotal();
           }

           Hook::applyFilter(
                          'render_placeholder_hits',
                          'hook_render_placeholder_hits'
           );
\end{DoxyVerb}


The first parameter of {\itshape \hyperlink{class_hook}{Hook}\-:apply\-Filter}, the hook filter, ends in 'hits' which correlates to the name of the placeholder. This hook filter is defined in '{\itshape \hyperlink{class_8template_8php}{class.\-template.\-php}}'. The second parameter is the name of the function to run when we render the placeholder.

You can iterate through multidimensional array data using \#each placeholders \begin{DoxyVerb}           // Bind the data like this
           $page->bind('fruits', array(
                          array(
                                         "name" => "apple",
                                         "color" => "red"
                          ), array(
                                         "name" => "banana",
                                         "color" => "yellow"
                          )
           ));

           // in the template
           {{ #each f in fruits }}
                          <p>I like {{ f.color }}, because my {{ f.name }} is {{ f.color }}.</p>
           {{ /each }}
\end{DoxyVerb}


\subsubsection*{Databases}

The database connection settings are defined in the $\ast$/include/global.php$\ast$ file. After the {\itshape L\-I\-V\-E\-\_\-\-U\-R\-L} is set in {\itshape \hyperlink{global_8php}{global.\-php}} the framework will detect which \hyperlink{class_d_b}{D\-B} to use based on the current U\-R\-L.

To get a database instance, use\-: \begin{DoxyVerb}           $db = DB::getInstance();
\end{DoxyVerb}


The \hyperlink{class_d_b}{D\-B} class is static and will automatically handle suppressing multiple instances.

\subsubsection*{Sending emails}

The \hyperlink{class_mailer}{Mailer} class simplifies sending emails by generating headers for you and using an easy to use object to clearly define your email. The \hyperlink{class_mailer}{Mailer} can send emails using an H\-T\-M\-L template or text. \begin{DoxyVerb}           $m = new Mailer();
           $m->addTo("test@test.com");
           $m->addFrom("from.me@test.com");
           $m->addSubject("This is a test, don't panic.");
           $m->fetchHTML("http://test.com/template.php");
           // OR
           $m->msgText("This is my message.")
           $m->send();
\end{DoxyVerb}


\subsubsection*{\hyperlink{class_c_s_v}{C\-S\-V}}

The \hyperlink{class_c_s_v}{C\-S\-V} class ensures that all records are properly escaped and allows you to easily manipulate data inside of a \hyperlink{class_c_s_v}{C\-S\-V} file. \begin{DoxyVerb}           $c = new CSV();
           $data = array(
                          'George',
                          'Washington'
           );
           $c->add($data);

           // Saves to the filesystem
           $c->save('presidents.csv');

           // OR

           // Sends the file to the browser, does not save to the filesystem
           $c->show();
\end{DoxyVerb}


\subsubsection*{Debugging}

The {\itshape \hyperlink{class_debug}{Debug}} static class allows you to debug on-\/screen, via email, or by logging to a database. \begin{DoxyVerb}           $db = DB::getInstance();
           Debug::out($db);
\end{DoxyVerb}


\subsubsection*{File System Database (\hyperlink{class_8fsdb_8php}{class.\-fsdb.\-php})}

Sometimes using a database is overkill. A simple solution is to use the {\itshape \hyperlink{class_f_s_d_b}{F\-S\-D\-B}}. It is very simple and does not allow complex queries, however it is fast, easy to use, and requires no setup, except checking that proper permissions are set. \begin{DoxyVerb}           $fruit = new stdClass();

           $fruit->name = "Apple";
           $fruit->color = "Red";
           $fruit->texture = "Crispy";
           $fruit->price = 0.50;

           $db = new FSDB();

           $db->insert('fruit', $fruit);
           $data = $db->select('fruit', 'name', 'Apple');
\end{DoxyVerb}


\subsubsection*{Country detection}

{\itshape Country detection} uses the {\itshape \hyperlink{class_f_s_d_b}{F\-S\-D\-B}} to do a quick lookup of the current country. \begin{DoxyVerb}           $i = new IP2CO();

           $countryCode = $i->getCountryCode($_SERVER['REMOTE_ADDR']);

           if ($countryCode != false) {
                          echo $countryCode;
           } else {
                          echo $_SERVER['REMOTE_ADDR'] . "(" . ip2long($_SERVER['REMOTE_ADDR']) . ") Not found in " . $i->getTable($_SERVER['REMOTE_ADDR']) . ".";
           }
\end{DoxyVerb}


\subsubsection*{Mobile detection}

Mobile detection is done by comparing the U\-A (user-\/agent) to a list of currently available mobile and tablet U\-A. \begin{DoxyVerb}           $md = new MobiDetect();

           if ($md->isMobile()) {
                          // goto mobile site
           }
\end{DoxyVerb}


\subsubsection*{\hyperlink{class_navigation}{Navigation}}

The navigation is generated from J\-S\-O\-N. It renders the J\-S\-O\-N into a unordered list with some classes added for the current active page. \begin{DoxyVerb}           // Add a placeholder in your template
           {{ TopNav }}

           // Create a php file in */modules/enabled/*
           require_once('include/class.navigation.php');

           // create a function to add to the *hook filter*
           function hook_render_placeholder_TopNav() {

                          // Page data is passed via JSON
                          $topNavData = '{
                                         "pages": [
                                                        {
                                                                       "title": "Nav 1",
                                                                       "link": "/nav1/"
                                                        }, {
                                                                       "title": "Nav 2",
                                                                       "link": "/nav2/",
                                                                       "pages": [
                                                                                      {
                                                                                                     "title": "Subnav 1",
                                                                                                     "link": "/downloads/fpo.pdf",
                                                                                                     "target": "_blank"
                                                                                      }
                                                                       ]
                                                        }
                                         ]
                          }';

                          $topNav = new Navigation($topNavData);
                          $topNav->setCurrent($_SERVER['SCRIPT_NAME']);

                          return $topNav->show();
           }

           Hook::applyFilter(
                          'render_placeholder_TopNav',
                          'hook_render_placeholder_TopNav'
           );\end{DoxyVerb}
 